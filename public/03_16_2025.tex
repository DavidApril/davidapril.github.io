\documentclass[letterpaper,11pt]{article}

\usepackage{latexsym}
\usepackage[empty]{fullpage}
\usepackage{titlesec}
\usepackage{marvosym}
\usepackage[usenames,dvipsnames]{color}
\usepackage{verbatim}
\usepackage{enumitem}
\usepackage[hidelinks]{hyperref}
\usepackage{fancyhdr}
\usepackage[english]{babel}
\usepackage{tabularx}
\input{glyphtounicode}


%----------FONT OPTIONS----------
% sans-serif
% \usepackage[sfdefault]{FiraSans}
% \usepackage[sfdefault]{roboto}
% \usepackage[sfdefault]{noto-sans}
% \usepackage[default]{sourcesanspro}

% serif
% \usepackage{CormorantGaramond}
% \usepackage{charter}


\pagestyle{fancy}
\fancyhf{} % clear all header and footer fields
\fancyfoot{}
\renewcommand{\headrulewidth}{0pt}
\renewcommand{\footrulewidth}{0pt}

% Adjust margins
\addtolength{\oddsidemargin}{-0.5in}
\addtolength{\evensidemargin}{-0.5in}
\addtolength{\textwidth}{1in}
\addtolength{\topmargin}{-.5in}
\addtolength{\textheight}{1.0in}

\urlstyle{same}

\raggedbottom
\raggedright
\setlength{\tabcolsep}{0in}

% Sections formatting
\titleformat{\section}{
  \vspace{-4pt}\scshape\raggedright\large
}{}{0em}{}[\color{black}\titlerule \vspace{-5pt}]

% Ensure that generate pdf is machine readable/ATS parsable
\pdfgentounicode=1

%-------------------------
% Custom commands
\newcommand{\resumeItem}[1]{
  \item\small{
    {#1 \vspace{-2pt}}
  }
}

\newcommand{\resumeSubheading}[4]{
  \vspace{-2pt}\item
    \begin{tabular*}{0.97\textwidth}[t]{l@{\extracolsep{\fill}}r}
      \textbf{#1} & #2 \\
      \textit{\small#3} & \textit{\small #4} \\
    \end{tabular*}\vspace{-7pt}
}

\newcommand{\resumeSubSubheading}[2]{
    \item
    \begin{tabular*}{0.97\textwidth}{l@{\extracolsep{\fill}}r}
      \textit{\small#1} & \textit{\small #2} \\
    \end{tabular*}\vspace{-7pt}
}

\newcommand{\resumeProjectHeading}[2]{
    \item
    \begin{tabular*}{0.97\textwidth}{l@{\extracolsep{\fill}}r}
      \small#1 & #2 \\
    \end{tabular*}\vspace{-7pt}
}

\newcommand{\resumeSubItem}[1]{\resumeItem{#1}\vspace{-4pt}}

\renewcommand\labelitemii{$\vcenter{\hbox{\tiny$\bullet$}}$}

\newcommand{\resumeSubHeadingListStart}{\begin{itemize}[leftmargin=0.15in, label={}]}
\newcommand{\resumeSubHeadingListEnd}{\end{itemize}}
\newcommand{\resumeItemListStart}{\begin{itemize}}
\newcommand{\resumeItemListEnd}{\end{itemize}\vspace{-5pt}}

%-------------------------------------------
%%%%%%  RESUME STARTS HERE  %%%%%%%%%%%%%%%%%%%%%%%%%%%%


\begin{document}

\begin{center}
    \textbf{\Huge \scshape Jesús D. Agámez} \\ \vspace{1pt}
    \small (+57) 3184926452 $|$ \href{mailto:jesusdavidagamez4@gmail.com}{\underline{jesusdavidagamez4@gmail.com}} $|$ 
    \href{https://linkedin.com/in/david-agamez}{\underline{linkedin.com/in/david-agamez}} $|$
    \href{https://github.com/DavidApril}{\underline{github.com/DavidApril}}
    \href{https://app.hackthebox.com/profile/1580898}{\underline{hackthebox/1580898}}
\end{center}

%-----------PROFESIONAL PROFILE-----------
\section{Perfil profesional}
    \par
    {Estudiante de Ingeniería de Software con 2 años de experiencia en el desarrollo especializado en tecnologías como JavaScript, TypeScript, y frameworks modernos como NestJS, ReactJS y NextJS. Poseo conocimientos en control de versiones con Git, diseño de bases de datos y servicios web, complementados con habilidades básicas en diseño web (HTML/CSS). Experiencia con el ciclo completo del desarrollo de software, desarrollo de pipelines (CI/CD) para automatización y Docker + Kubernetes para contenerización.}


%-----------EXPERIENCE-----------
\section{Experiencia}
  \resumeSubHeadingListStart
    \resumeSubheading
      {Ingeniero de Software}{Junio 2023 -- Marzo 2025}
      {DevZeros S.A.S}{Valledupar, Colombia}
      \
      \par{Desarrollador de software, principalmente con ReactJs, React-Native NextJs y NestJS  + TypeScript. Entre mis
      responsabilidades:}
    
      \resumeItemListStart
        \resumeItem{Participar en la planificación y estimación de tareas y proyectos pequeños a mediano alcance.}
        \resumeItem{Depurar y resolver problemas de rendimiento en aplicaciones legacy.}
        \resumeItem{Seguir procedimientos operativos estándar y mejorar prácticas en el desarrollo mediante herramientas de gestión de proyectos como Jira, Azure DevOps y Github Projects.}
        \resumeItem{Realizar pruebas unitarias y revisiones de código para asegurar entregables de alta calidad.}
        \resumeItem{Automatizar procesos de integración y despliegue continuo (CI/CD) utilizando herramientas como Github Actions y Jenkins.}
        \resumeItem{Colaborar con equipos multidisciplinarios para alinear los requerimientos técnicos con las necesidades del negocio.}
        \resumeItem{Desarrollar e implementar soluciones que sigan principios arquitectónicos como Clean Architecture.}
        \resumeItem{Documentar procesos, configuraciones y desarrollos para asegurar la continuidad del conocimiento técnico.}
      \resumeItemListEnd
    \resumeSubHeadingListEnd

%-----------EDUCATION-----------
\section{Educación}
  \resumeSubHeadingListStart
  
    \resumeSubheading
      {Corporación Minuto de Dios}{Colombia}
      {Ingeniero de Software}{Enero. 2023 -- Dic 2026}
    \
    \par{Estudiante de 5to semestre de ingeniería de software donde he adquirido sólidos conocimientos en:}
    \resumeItemListStart
        \resumeItem{Sistemas operativos, Gestión y administración de la memoria}
        \resumeItem{Concurrencia}
        \resumeItem{Analisís y diseño de software}
        \resumeItem{Estructura de datos y análisis de algoritmos}
        \resumeItem{Infraestructura TI}
        \resumeItem{Sistemas de gestión de base de datos}
        \resumeItem{Desarrollo de Software Orientado a la web}
   \resumeItemListEnd
   
  \resumeSubheading
  {Comfacesar Rodolfo Campo Soto}{Valledupar, Colombia}
  {Bachiller académico}{Enero. 2012 -- Dic 2019}
  \
  \par{Mención de honor por superación Comfacesarence}
  \resumeItemListStart
    \resumeItem{Lógica de programación}
    \resumeItem{Diseño básico de circuitos electrónicos con Arduino}
    \resumeItem{Cálculo diferencial e integral}
  \resumeItemListEnd
   
  \resumeSubheading
  {Instecom}{Valledupar, Colombia}
  {Mecánico en reparación y mantenimiento de equipos electrónicos}{Dic 2019}
  \
  \par{Realicé mantenimiento preventivo y correctivo en máquinas y sistemas mecánicos. Llevando a cabo inspecciones para identificar problemas y anomalías operativas}
  \resumeItemListStart
    \resumeItem{Limpieza profunda de hardware}
    \resumeItem{Ensamblaje y desensamblaje de equipos electrónicos}
  \resumeItemListEnd


  \resumeSubHeadingListEnd

\section{Cursos}
  \resumeSubHeadingListStart

  \resumeSubheading
  {Curso completo de Hacking Ético y Ciberseguridad}{Udemy}
  {Analista de seguridad, Santiago Hernández}{}
  \
  \par{Conocimiento neceario para realizar auditorias de seguridad}
  \resumeSubheading
  {Curso completo de Linux}{Udemy}
  {Administrador de sistemas Linux, Santiago Hernández}{}
  \
  \par{Fundamentos necesarios para la administración de sistemas operativos Linux}
  \resumeSubheading
  {NestJS + Microservicios: Aplicaciones escalables y modulares}{Udemy}
  {Desarrollo backend, Fernando Herrera}{}
  \
  \par{NATS, Webhooks, CI/CD, Git submodules, Gateways, Docker, Kubernetes, GCloud, PosstgreSQL, Mongo, SQLite}
  \resumeSubheading
  {NesjJS: Desarrollo backend escalable con Node}{Udemy}
  {Desarrollo backend, Fernando Herrera}{}
  \
  \par{Storybook, Github Actions, publicar NPM, TypeScript, patrones de componentes, PWA, Formik, formularios dinámicos, etc. }
  \resumeSubheading
  {NodeJS: De cero a experto}{Udemy}
  {Desarrollo backend, Fernando Herrera}{}
  \
  \par{Clean Architecture, DDD, WebHooks, WebSockets, Tareas automáticas, Despliegues, TypeScript, Edge, Testing}

  \par{REST, TypeScript, Websockets, Autenticación, Authorización, Docker, Mongo, Postgres, TypeOrm}
  \resumeSubheading
  {React PRO: Lleva tus bases al siguiente nivel}{Udemy}
  {Desarrollo frontend, Fernando Herrera}{}
  \
  \par{Storybook, Github Actions, publicar NPM, TypeScript, patrones de componentes, PWA, Formik, formularios dinámicos, etc.}
  \resumeSubheading
  {React: De cero a experto}{Udemy}
  {Desarrollo frontend, Fernando Herrera}{}
  \
  \par{Context API, MERN, Hooks, Firestore, JWT, Testing, Autenticaciones, Despliegues, CRUD, Logs, MUI, Multiple Routers, etc.}

  \resumeSubHeadingListEnd

%-----------PROJECTS-----------
\section{Proyectos}
    \resumeSubHeadingListStart
      \resumeProjectHeading
          {\textbf{Microservicio de autenticación} $|$ \emph{TypeScript, NestJS, PostgreSQL, TypeORM, Docker}}{Dic 2024 -- Present}
          \resumeItemListStart
          \resumeItem{Microservicio de Authenticación y autorización desarrollado en NestJS}
          \resumeItem{Compilación con SWC: Utiliza Speedy web copiler para transpilar el código}
          \resumeItem{Docker para orquestación}
          \resumeItem{Gestión de usuarios, sessión, roles y permisos}
          \resumeItem{Husky para comprobación y confirmación de commits}
          \resumeItem{CI/CD para un registro de cambios y lanzamientos automáticos}
          \resumeItemListEnd
      \resumeSubHeadingListEnd

%-----------PROGRAMMING SKILLS-----------
\section{Competencias técnicas}
 \begin{itemize}[leftmargin=0.15in, label={}]
    \small{\item{
     \textbf{Lenguajes}{: Bash, Python, C/C++, MySQL, JavaScript, TypeScript} \\
     \textbf{Frameworks}{: React, NextJS, AngularJS NestJS, Node.js, Flask, JUnit, FastAPI} \\
     \textbf{Herramients de desarrollo}{: Linux, Git, GitHub, Docker, Google Cloud Platform, Neovim, Zed, Visual Studio, PyCharm, IntelliJ} \\
     \textbf{Librerías}{: Motion, Cheerio, pandas, NumPy, Matplotlib, Express, Axios, Formik}
    }}
 \end{itemize}

%-----------LANGUAGES------------
 \section{Idiomas}
 \begin{itemize}[leftmargin=0.15in, label={}]
   \small{\item{
       \textbf{Español}{: Nativo} \\
       \textbf{Inglés}{: B2} \\
   }}
\end{itemize}

%-------------------------------------------
\end{document}
